\documentclass[aspectratio=169]{beamer}
\usetheme{Boadilla}
\useoutertheme{infolines}
\setbeamertemplate{navigation symbols}{}

\usepackage{fontspec}
\usepackage{unicode-math}
\usepackage[Latin,Greek]{ucharclasses}
\usepackage{amsmath}
\usepackage{proof}
\usepackage[backend=biber]{biblatex}
\addbibresource{references.bib}
\usepackage{tikz}
\usetikzlibrary{cd}
\usepackage{adjustbox}

\makeatletter
\def\th@remarkstyle{
    \normalfont
    \setbeamercolor{block title example}{bg=orange!40,fg=orange}
    \setbeamercolor{block body example}{bg=orange!20,fg=black}
    \def\inserttheoremblockenv{exampleblock}
  }
\makeatother
\theoremstyle{remarkstyle}
\newtheorem*{remark}{Remark}

\title{Co-Debruijn: Everybody's Got To Be Somewhere$^{\text{\cite{codebruijn}}}$}
\subtitle{From Debruijn to co-Debruijn using Category Theory}
\institute[Uni Freiburg]{Chair of Programming Languages, University of
  Freiburg}
\author{Marius Weidner}

\begin{document}

\begin{frame}
  \titlepage{}
\end{frame}

\begin{frame}{Outline}
  \tableofcontents
\end{frame}

\section{Getting Started: Scopes and Binders Categorically}
\subsection{The Category of Scopes}

\begin{frame}[fragile]
  \frametitle{The Category of Scopes: $Δ_+^X$}
  \begin{definition}
    Let $Δ_+^X$ be the category of scopes.
    \begin{itemize}
      \item Objects: $\bar{x}, \bar{y}, \bar{s} ∈ |Δ_+^X| = X^*$
      \item Morphisms: $f, g ∈ Δ_+^X(\bar{x}, \bar{y})$ for $\bar{x}, \bar{y} ∈ X^*$ are inductively defined:
    \end{itemize}
    \begin{columns}
      \begin{column}{0.04\textwidth}
      \end{column}
      \begin{column}{0.2\textwidth}
        \begin{center}
          \infer[·]{
            ε ⊑ ε
          }{}
        \end{center}
      \end{column}
      \begin{column}{0.2\textwidth}
        \begin{center}
          \infer[1]{
            \bar{x}x ⊑ \bar{y}x
          }{
            \bar{x} ⊑ \bar{y}
          }
        \end{center}
      \end{column}
      \begin{column}{0.2\textwidth}
        \begin{center}
          \infer[0]{
            \bar{x} ⊑ \bar{y}y
          }{
            \bar{x} ⊑ \ \bar{y}
          }
        \end{center}
      \end{column}
      \begin{column}{0.3\textwidth}
      \end{column}
    \end{columns}
  \end{definition}
  \begin{corollary}
    The initial object of the $Δ_+^X$ category is the empty scope $ε$ with the $\bar{0}$ as the unique morphism. 
  \end{corollary}
  \begin{remark}
    Morphisms in $Δ_+^X$ can be represented by \emph{bit vectors} $\bar{b} ∈ \{0, 1\}^*$ with one bit per variable of the target scope telling whether it has been mapped to or skipped by the source scope.
  \end{remark}
\end{frame}

\begin{frame}[fragile]
  \frametitle{Objects \& Morphisms in $Δ_+^⊤$}
  \begin{example}
    Let $X = ⊤$ (where $⊤$ is the set with exactly one element $⟨⟩$).\\Thus, Objects $\bar{x} ∈ X^*$ represents numbers. 
    \begin{columns}
      \begin{column}{0.6\textwidth}
        \begin{center}
          \adjustbox{scale=0.6}{
            \begin{tikzcd}
              3 \arrow[rr, "10101"]  &         & 5 &   &   &   &   &   &   \\
                                     &         &   &   &   &   &   &   &   \\
              • \arrow[rr, no head]  &         & • &   &   &   &   &   & 1 \\
                                     &         & ∘ &   &   &   &   & 0 &   \\
              • \arrow[rr, no head]  &         & • &   &   &   & 1 &   &   \\
                                     &         & ∘ &   &   & 0 &   &   &   \\
              • \arrow[rr, no head]  &         & • &   & 1 &   &   &   &   \\
                                     &         &   & · &   &   &   &   &  
            \end{tikzcd}
          }
        \end{center}
      \end{column}
      \begin{column}{0.4\textwidth}
        \begin{center}
          \adjustbox{scale=1.3}{
            \begin{tikzcd}
              3 \arrow[d, "10101"', bend right] \arrow[d, dashed, bend left] \arrow[rd, dotted, bend left] \arrow[d, dashed, bend left=60] &    \\
              5                                                                                                                             & {}
            \end{tikzcd}
          }
        \end{center}
      \end{column}
    \end{columns}
  \end{example}
\end{frame}

\begin{frame}[fragile]
  \frametitle{Identity and Composition in $Δ_+^⊤$}
  \begin{columns}
    \begin{column}{0.20\textwidth}
      \begin{example}
        \begin{center}
          \adjustbox{scale=0.8}{
            \begin{tikzcd}
              5 \arrow[rr, "id \ 5" = 11111]    &  & 5 \\
                                    &  &   \\
              • \arrow[rr, no head] &  & • \\
              • \arrow[rr, no head] &  & • \\
              • \arrow[rr, no head] &  & • \\
              • \arrow[rr, no head] &  & • \\
              • \arrow[rr, no head] &  & • 
            \end{tikzcd}
          }
        \end{center}
      \end{example}
    \end{column}
    \begin{column}{0.75\textwidth}
      \begin{example}
        \adjustbox{scale=0.8}{
          \begin{tikzcd}
            2 \arrow[rr, "101"]  &  & 3 &  & 3 \arrow[rr, "10101"]&  & 5 &  & 2 \arrow[rr, "10001"]   &  & 5 \\
                                  &  &   &  &                       &  &   &  &                       &  &   \\
            • \arrow[rr, no head] &  & • &  & • \arrow[rr, no head] &  & • &  & • \arrow[rr, no head] &  & • \\
                                  &  &   &  &                       &  & ∘ &  &                       &  & ∘ \\
                                  &  & ∘ & ; & • \arrow[rr, no head] &  & • & =  &                       &  & ∘ \\
                                  &  &   &  &                       &  & ∘ &  &                       &  & ∘ \\
            • \arrow[rr, no head] &  & • &  & • \arrow[rr, no head] &  & • &  & • \arrow[rr, no head] &  & •
          \end{tikzcd}
        }
      \end{example}
    \end{column}
  \end{columns}
\end{frame}

\begin{frame}[fragile]
  \begin{columns}
    \begin{column}{0.35\textwidth}
      \large{\color{blue} $Δ_+^X$ is in Fact a Category}
      \begin{lemma}
        \begin{small}
          In $Δ_+^X$ every object $\bar{x} ∈ X^*$ has an identity morphism, 
          i.e. we can construct an identity morphism for $\bar{x}$ using the inference rules.
        \end{small}
      \end{lemma}
      \begin{proof}
        \begin{small}
        $id \ \ \ \ \ \ : \ (\bar{x} : X^*) → \bar{x} ⊑ \bar{x}$\\
        $id \ ε \ \ \ = \ ·$\\
        $id \ \bar{x} x \ = \ ($id $ \bar{x})1$
        \end{small}
      \end{proof}
      \begin{corollary}
        \begin{small}
        $id-l \ \, : \ id ; f = f$ \\
        $id-r \ : \ f;id = f$
        \end{small}
      \end{corollary}  
    \end{column}
    \begin{column}{0.6\textwidth}
      \begin{lemma}
        \begin{small}
        In $Δ_+^X$ two morphisms $f : \bar{x} ⊑ \bar{y}$ and $g : \bar{y} ⊑ \bar{z}$ 
        compose to a morphism $f;g : \bar{x} ⊑ \bar{z}$, 
        i.e. we can construct a morphism $f;g$ from $f$ and $g$ using the inference rules.
        \end{small}
      \end{lemma}
      \begin{proof}
        \begin{small}
        $\_;\_ \ \ \ \ \ \ : \ \bar{x} ⊑ \bar{y} → \bar{y} ⊑ \bar{z} → \bar{x} ⊑ \bar{z}$\\
        $·  \ \ \, ; \ · \ \ \ \ = \ ·$\\
        $f 1  \ ; \ g 1 \ = \ (f;g)1$\\
        $f 0 \ ; \ g 1  \ = \ (f;g)0$\\
        $f \ \ \, ; \ g 0 \ = \ (f;g)0$
        \end{small}
      \end{proof}
      \begin{corollary}
        \begin{small}
        assoc \ \ \ \ : \ $f;(g;h) = (f;g);h$\\
        antisym \ : \ $(f : \bar{x} ⊑ \bar{y}) → (g : \bar{y} ⊑ \bar{z}) → \bar{x} = \bar{y} \land f = g = $ id $\bar{x}$
        \end{small}
      \end{corollary}  
    \end{column}
  \end{columns}
\end{frame}


\subsection{Intrinsically Scoped Debruijn Syntax}

\begin{frame}[fragile]
  \frametitle{Intrinsically Scoped Debruijn Syntax via $Δ_+^⊤$}
  \begin{definition}
    Let $Tm : |Δ_+^⊤| → Set$ be inductively defined: \\
    \begin{columns}
      \begin{column}{0.3\textwidth}
        \begin{center}
          \infer[\#]{
            Tm \ \bar{x}
          }{
            ⟨⟩ ⊑ \bar{x}
          }
        \end{center}
      \end{column}
      \begin{column}{0.3\textwidth}
        \begin{center}
          \infer[\$]{
            Tm \ \bar{x}
          }{
            Tm \ \bar{x} &  
            Tm \ \bar{x}
          }
        \end{center}
      \end{column}
      \begin{column}{0.3\textwidth}
        \begin{center}
          \infer[λ]{
            Tm \ \bar{x}
          }{
            Tm \ \bar{x}⟨⟩
          }
        \end{center}
      \end{column}
    \end{columns}
  \end{definition}
  \begin{example}
    
  \end{example}
\end{frame}

\begin{frame}[fragile]
  \frametitle{Lifting Scope Indexed Terms using Composition in $Δ_+^⊤$}
  \begin{lemma}
    Given an intrinsically scoped term $t ∈ Tm \ \bar{x}$ we can \emph{lift} $t$ to a $Tm \ \bar{y}$,
    if there exists a morphism $\bar{x} ⊑ \bar{y} ∈ Δ_+^⊤(\bar{x}, \bar{y})$, 
    i.e. $\bar{x}$ is a subscope of $\bar{y}$.
  \end{lemma}
  \begin{proof}
    $\_↑\_ \ \ \ \ \ \ \ \ \ \ \,  : \ Tm \ \bar{x} → \bar{x} ⊑ \bar{y} → Tm \ \bar{y}$\\
    $(\# \ v) \ \ \  \, ↑ f \ = \ \# \ (v ; f)$\\
    $(t₁ \ \$ \ t₂) ↑ f \ = \ (t₁ ↑ f) \ \$ \ (t₂ ↑ f)$\\
    $(λ \ t) \ \ \ \ \, ↑ f  \ = \ λ \ (t ↑ S f)$
  \end{proof}
\end{frame}


\section{Going Further: Stronger Intrinsic Invariants over Scopes and Binders}
\subsection{The Slice Category of Subscopes}

\begin{frame}[fragile]
  \frametitle{The Slice Category of Subscopes: $Δ_+^X∖\bar{s}$}
  \begin{definition}
    \begin{columns}
      \begin{column}{0.68\textwidth}
        Let $Δ_+^X∖\bar{s}$ be the category of subscopes for a given $\bar{s} ∈ X^*$. 
        \begin{itemize}
          \item Objects: $\bar{b}, (\bar{x}, f) ∈ |Δ_+^X∖\bar{s}| = \left(\bar{x} : X^* × Δ_+^X(\bar{x}, \bar{s}) \right)$
          \item Morphisms: $h ∈ [Δ_+^X∖\bar{s}]((\bar{x}, f), (\bar{y}, g))$ such that $f = h;g$ 
        \end{itemize}
      \end{column}
      \begin{column}{0.3\textwidth}
        \adjustbox{scale=1}{
          \begin{tikzcd}
            \bar{x} \arrow[rdd, "f"', dotted] \arrow[rr, "h"] &   & \bar{y} \arrow[ldd, "g", dotted] \\
                                                        &   &                            \\
                                                        & \bar{s} &                           
          \end{tikzcd}
        }
      \end{column}
    \end{columns}
    
  \end{definition}
  \begin{corollary}
    The initial object of the $Δ_+^X∖\bar{s}$ category is the empty subscope $(ε, \bar{0})$. 
  \end{corollary}
  \begin{remark}
    Objects in $Δ_+^X∖\bar{s}$ can be represented by \emph{bit vectors} $\bar{b} ∈ \{0, 1\}^*$ with one bit per variable of scope $\bar{s}$, telling whether it has been selected.
  \end{remark}
\end{frame}

\begin{frame}[fragile]
  \frametitle{Objects \& Morphisms in $Δ_+^T∖5$}
  \begin{example}
    \begin{columns}
      \begin{column}{0.3\textwidth}
        \adjustbox{scale=1}{
          \begin{tikzcd}
            3 \arrow[rdd, "01110"', dotted] \arrow[rr, "0111"]  &   & 4 \arrow[ldd, "11110", dotted] \\
                                                                &   &                                 \\
                                                                & 5 &                                
            \end{tikzcd}
        }
      \end{column}
      \begin{column}{0.65\textwidth}
        \adjustbox{scale=0.6}{
          \begin{tikzcd}
            3 \arrow[rr, "0111"]  &  & 4 &   & 4 \arrow[rr, "11110"] &  & 5 &  & 3 \arrow[rr, "01110"] &  & 5 \\
                                  &  &   &   &                        &  &   &  &                        &  &   \\
                                  &  & ∘ &   & • \arrow[rr, no head]  &  & • &  &                        &  & ∘ \\
            • \arrow[rr, no head] &  & • &   & • \arrow[rr, no head]  &  & • &  & • \arrow[rr, no head]           &  & • \\
            • \arrow[rr, no head] &  & • & ; & • \arrow[rr, no head]  &  & • & = & • \arrow[rr, no head]  & & • \\
            • \arrow[rr, no head] &  & • &   & • \arrow[rr, no head]  &  & • &  & • \arrow[rr, no head]  &  & • \\
                                  &  &   &   &                        &  & ∘ &  &                        &  & ∘
            \end{tikzcd}
        }
      \end{column}
    \end{columns}
    Alternatively:
    \begin{itemize}
    \item \adjustbox{scale=1}{
      \begin{tikzcd}
        {(3, 01110)} \arrow[rr, "0111"] &  & {(4, 11110)}
      \end{tikzcd}
    }
    \item \adjustbox{scale=1}{
      \begin{tikzcd}
        {01110} \arrow[rr, "0111"] &  & {11110}
      \end{tikzcd}
    }
    \end{itemize}
  \end{example}
\end{frame}

\begin{frame}[fragile]
  \frametitle{Coproducts in $Δ_+^X∖\bar{s}$}
  \begin{theorem}
    Objects in the slice category $\bar{b}₁, \bar{b}₂ ∈ |Δ_+^X∖\bar{s}|$ have a coproduct object $\bar{b}₁ + \bar{b}₂$ if there exist morphisms $l ∈ [Δ_+^X∖\bar{s}](\bar{b}₁, \bar{b}₁ + \bar{b}₂)$ and $r ∈ [Δ_+^X∖\bar{s}](\bar{b}₂, \bar{b}₁ + \bar{b}₂)$. Then for every $f ∈ [Δ_+^X∖\bar{s}](\bar{b}₁, \bar{b}_3)$ and $g ∈ [Δ_+^X∖\bar{s}](\bar{b}₂, \bar{b}_3)$, there exists a unqiue $h ∈ [Δ_+^X∖\bar{s}](\bar{b}₁ + \bar{b}₂, \bar{b}_3)$ such that $f = l;h$ and $g = r;h$.
  \end{theorem}
  \begin{columns}
    \begin{column}{0.5\textwidth}
      \begin{example}
          \adjustbox{scale=0.76}{
            \begin{tikzcd}
              01100 \arrow[rrrrrd, "0110"] \arrow[rrd, "011"'] &  &                           &  &  &       \\
                                                               &  & 11100 \arrow[rrr, "1110"] &  &  & 11110 \\
              11000 \arrow[rrrrru, "1100"'] \arrow[rru, "110"] &  &                           &  &  &      
              \end{tikzcd}
          }
      \end{example}
    \end{column}
    \begin{column}{0.46\textwidth}
      \begin{remark}
        The coproduct $\bar{b}₁ + \bar{b}₂$ of two subscopes $\bar{b}₁, \bar{b₂}$ 
        corresponds to the minimal subscope covering both $\bar{b}₁$ and $\bar{b₂}$. 
        The coproduct $\bar{b}₁ + \bar{b}₂$ can be computed by pointswise disjunction of $\bar{b}₁$ and $\bar{b}₂$. 
      \end{remark}
    \end{column}
  \end{columns}
\end{frame}

\subsection{A Monad Over Sets Indexed by Scopes}

\begin{frame}[fragile]
  \frametitle{Category of Sets Indexed by Scopes}
  \begin{definition}
    Let $Set_X$ be the category of sets indexed by scopes $\bar{x} ∈ X^*$.
    \begin{itemize}
      \item Objects: $T, S ∈ |Set_X| = X^* → Set = \bar{X}$
      \item Morphisms: $f_⋅ ∈ Set_X(T, S) = (\bar{x}∈X^*) → T(\bar{x}) → S(\bar{x}) = T \stackrel{⋅}{→} S$
    \end{itemize}
  \end{definition}
  \begin{definition}
    Let $\_⇑\_ : \bar{X} → \bar{X} = (T, \bar{x}) ↦ (T(\bar{s}) × \bar{s} ⊑ \bar{x})$.\\
    We define $Ref : Set_X \stackrel{⋅}{→} Set_X$ to be the endofunctor induced by the mapping
    \begin{itemize}
      \item $Ref(T) = \bar{x} ↦ T ⇑ \bar{x} ∈ \bar{X}$
      \item{$Ref(f_⋅) = (t, h) ↦ (f_⋅(t) , h) ∈ T \stackrel{⋅}{→} S$}
    \end{itemize}
  \end{definition}
  \begin{remark}
    The set $T⇑\bar{x}$ packs an set $T ∈ \bar{X}$ indexed by $\bar{x} ∈ X^*$ applied to a subscope $\bar{s}$ of $\bar{x}$, together with a selection $h ∈ |Δ_+^X∖\bar{x}|$ of the variables of $T$.
  \end{remark}
\end{frame}


% \begin{frame}[fragile]
%   \frametitle{$Ref$ by example}
%   \begin{example}
%     \adjustbox{scale=1.05}{    
%       \begin{tikzcd}
%         & {} \arrow[rrrr, "Ref", dashed, bend left] &   &  &                                                                                & {} &   \\
%         Tm  \arrow[rr, "f_⋅"] & {} \arrow[r] & T &  & {\bar{x} ↦ (Tm \ \bar{s}, \bar{s} ⊑ \bar{x})} \arrow[rr, "{(tm, h) ↦ (f_⋅ \ tm, h)}"] &    & {\bar{x} ↦ (T \ \bar{s}, \bar{s} ⊑ \bar{x})}
%       \end{tikzcd}
%     }
%   \end{example}
% \end{frame}

\begin{frame}[fragile]
  \frametitle{$Δ_+^X$ makes $Ref$ a Monad!}
  \begin{theorem}
    The functor $Ref : Set_X → Set_X$ gives rise to a monad with the two natural transformations
    \begin{itemize}
      \item $unit : Id(T) \stackrel{⋅}{→} Ref(T) = t ↦ (t, id)$ 
      \item $mult : Ref(Ref(T)) \stackrel{⋅}{→} Ref(T) = ((t, h₁) h₂) ↦ (t, h₁;h₂)$
    \end{itemize}
  \end{theorem}
  \begin{example}
      \begin{tikzcd}
        Tm \arrow[rr, "unit", dashed] &  & \bar{x} ↦ (Tm \ \bar{x}, \bar{x} ⊑ \bar{x}) \arrow[lld, "Ref"', dashed] \\
        \bar{y} ↦ ([\bar{x} ↦ (Tm \ \bar{x}, \bar{x} ⊑ \bar{x})] \bar{s}, \bar{s} ⊑ \bar{y}) \arrow[rr, "mult", dashed]  &  & \bar{y} ↦ (Tm \ \bar{s}, \bar{s} ⊑ \bar{y})                        
      \end{tikzcd}
  \end{example}
\end{frame}

\begin{frame}[fragile]
  \frametitle{The Notion of Relevant Pairs}
  \begin{definition}
    Let $\_×_R\_ : \bar{X} → \bar{X} → \bar{X}  = (T, S, \bar{x}) ↦ ((\_, \bar{b}₁ : T ⇑ \bar{x}) × (\_, \bar{b}₂ : S ⇑ \bar{x}) × ∃\bar{b}(\bar{b} ≡ \bar{b}₁ + \bar{b}₂))$. \\ We write $\_,_R\_ : T ⇑ \bar{x} → S ⇑ \bar{x} → (T ×_R S) ⇑ \bar{x} = ((\_, \bar{b}₁), (\_, \bar{b}₂)) ↦  (L(\bar{b}₁+\bar{b}₂) , R(\bar{b}₁+\bar{b}₂) , \bar{b}₁+\bar{b}₂)$ to construct a $T×_R S$. \\
  \end{definition}
  \begin{example}
    We'll look at $Tm$

    Let $.. #1 .. #2 .. : Tm 5$
  \end{example}
\end{frame}

\subsection{Monoidal Structure of Scopes}

\subsection{Intrinsically Scopes co-Debruijn Syntax}

\subsection{Translating Debruijn Syntax to Codebruijn Syntax}


\section{Wrapping Up: What I've (Not) Told You}
\subsection{This Is Actually an Agda Paper (!)}
\subsection{Abstracting over Arbitrary Syntaxes}

\begin{frame}[fragile]
  \frametitle{References}
  \nocite{catsandtypes}
  \printbibliography{}
\end{frame}

\end{document}